\documentclass[11pt]{article}

\usepackage{geometry}
\usepackage{amsmath, amsthm, amssymb, bbm}
\usepackage{hyperref}
\usepackage{setspace}
\usepackage{xcolor}
\usepackage{natbib}
\usepackage{tikz}
\usetikzlibrary{decorations.pathreplacing,calc}

% Define colors for emphasis
\definecolor{mAlert}{RGB}{235, 129, 27}
\newcommand{\alert}[1]{\textcolor{mAlert}{#1}}

\newtheorem{assumption}{Assumption}
\newtheorem{proposition}{Proposition}
\newtheorem{lemma}{Lemma}
\newtheorem{corollary}{Corollary}

\title{AI, Headquarters and Guijie}
\author{Zehao Zhang}
% \date{\today}

\setstretch{1.5}

\begin{document}

\maketitle

\section{Model}

Two players: a headquarter (P) and a guijie (A). P initiates a campaign to boost sales. A successful campaign first depends on the market condition $\theta \in \{0, 1\}$, which P does not know well. With probability $q \in (0, 1)$, the campaign may be initiated at a time when the market is actually bad ($\theta = 0$). With probability $1-q$, the campaign is initiated at a correct time, i.e., $\theta = 1$. Guijie observes a noisy signal $s$ with $\Pr(s=1\mid \theta=1)=1$ and $\Pr(s=1\mid \theta=0)=p\in(0,1)$.

The second component of success is the guijie's effort $e \in \{0, 1\}$. Guijie bears a cost of $c$ if $e = 1$, no cost if $e = 0$.

The final outcome of the campaign is $y \in \{0, 1\}$, and $y = \theta e$. P receives a payoff of $\Pi_P$ if $y = 1$, and zero otherwise. Guijie does not profit directly from the campaign. To incentivize guijie to work, P offers a bonus $b$ to A if $y = 1$. Both P and A are risk neutral.

The timing is as follows: (i) P initiates the campaign and signs the bonus contract $b$ with A. (ii) Market condition $\theta$ realizes. Guijie observes $s\mid \theta$ and decides whether to exert effort $e$. (iii) P and A observe $y$ and receive their payoffs.

\section{Analysis}
Suppose $\Pi_P$ large enough. Upon observing $s=1$, the posterior of $\theta$ is $\Pr(\theta=1\mid s=1)=\frac{1-q}{1-q+pq}$. A chooses $e = 1$ iff $\frac{1-q}{1-q+pq} b - c \geq 0$. 

If A has a relatively poor signal ($p$ large), the bonus $b$ must be large enough to compensate the effort cost spent on a bad market. If the advent of AI improves the signal of A (i.e., a lower $p$), the bonus $b$ can be made smaller.

Note that $\frac{1-q}{1-q+pq}$ is also decreasing in $q$.  As a result, with the possible aid of AI, if P is more likely to initiate the campaign at a good time, a lower bonus $b$ will be sufficient to incentivize A to work.




\end{document}